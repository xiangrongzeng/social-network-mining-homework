% chose the documnet class as article
\documentclass{article}

% basic setteings
\topmargin=-0.45in
\evensidemargin=0in
\oddsidemargin=0in
\textwidth=6.5in
\textheight=9.0in
\headsep=0.25in

% create a new command 
\newcommand{\hwtitle}{Graph Essentials:Homework\ \#1}
\newcommand{\student}{Zeng Xiangrong}
\newcommand{\teacher}{Professor Hao Wang}

% this package is used for setting header and footer
\usepackage{fancyhdr}
\pagestyle{fancy}
% this package enable the chinese character
\usepackage{CJK}
% this package enable the hyplink
\usepackage{hyperref}

\begin{document}
% set the title page
\begin{titlepage}
\begin{center}
    \textbf{}
    \\[4.5cm]
    \textbf{\huge \hwtitle}
    \\[0.3cm]
    \textnormal{Due on}
    \today
    \\[0.4cm]
    \emph{\teacher}
    \\[8cm]
    \textbf{\Large \student}
\end{center}
\end{titlepage}

% setting the header and footer
\fancyhf{} % clear the header and footer
\lhead{\student}
\chead{\hwtitle}
\rhead{Problem \thesection}
\renewcommand{\footrulewidth}{0.4pt}
\cfoot{\thepage}

% setting the section style
\setcounter{section}{0} % set the counter start from 0
% set the section style to "problem 1" and increase correctly
\renewcommand{\section}{
    \stepcounter{section}
    \begin{flushleft}
        \raggedright \Large\textbf{Problem \thesection\\}
    \end{flushleft}}
\section
% put the main text of every problem here
Proof: Lemma2. In any directed graph, the summation of in-degrees is equal to the summation of out-degree.
\subsection*{Solution}
    One edge must comes from a node and end in a node. 
    So when a node has an out-degree,
    that means there is an edge,
    and the edge must have an end,
    that's the in-degree of an node.
    So,
    whenever there is an out-degree, 
    there must exist an in-degree in the graph.
    So,
    for the graph, 
    the in-degree and out-degree must the same.
\clearpage

\section
% put the main text of every problem here
Give an efficient bridge detection algorithm.
\subsection*{Solution}
\begin{CJK}{UTF8}{song}
    假如我们在DFS时访问到了u点,
    此时图被u点分成两部分,
    一部分是已经访问过的点,
    另一部分是还没有访问的点。\\
    如果u是割点,
    那么未访问过的点中至少有一个点在不经过u点的情况下无法访问那些已经访问过的点。
    即,
    对未访问过的点v进行DFS,
    且不经过u点,
    若能够访问到已访问过的点,
    则u不是割点,
    若不能访问到已访问过的点,
    则u是割点。\\\\
    参考资料:\url{http://blog.csdn.net/wtyvhreal/article/details/43530613}
\end{CJK}
\end{document}
